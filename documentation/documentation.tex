\documentclass[12pt, a4paper]{article}
\usepackage{hyperref}
\usepackage{lmodern, cmap}
\usepackage[T1]{fontenc}
\usepackage[czech]{babel}
\usepackage[utf8]{inputenc}
\usepackage{graphicx, array, blindtext}
\usepackage{ae}
\usepackage[tc]{titlepic}
\usepackage{amsmath}
\usepackage{enumitem}

\usepackage{courier}
\usepackage{listings}
\usepackage{color}
\definecolor{mygreen}{rgb}{0,0.6,0}
\definecolor{mygray}{rgb}{0.5,0.5,0.5}
\definecolor{mymauve}{rgb}{0.58,0,0.82}

\lstset{ %
backgroundcolor=\color{white}, % choose the background color
basicstyle=\footnotesize\ttfamily,	% size of fonts used for the code
breaklines=true, % automatic line breaking only at whitespace
captionpos=b, % sets the caption-position to bottom
commentstyle=\color{mygreen}, % comment style
escapeinside={\%*}{*)}, % if you want to add LaTeX within your code
keywordstyle=\color{blue}, % keyword style
stringstyle=\color{mymauve}, % string literal style
} 

\begin{document}

% nove makro pro automaticky popis a cislovani obrazku
\newcommand{\obrazek}[1]{Obrázek \ref{#1}}
% nove makro pro automaticky piopis a cislovani tabulek
\newcommand{\tabulka}[1]{Tabulka \ref{#1}}

\begin{titlepage}
\begin{center}
\includegraphics[width=0.65\textwidth]{fav.png} \\[0.5cm]  %logo
\textsc{\Large KATEDRA INFORMATIKY A VÝPOČETNÍ TECHNIKY }\\[1cm]

\textsc{\Large Semestrální práce z předmětu KIV/WEB}\\[1cm]

\textsc{\Large Webové stránky konference}\\[0.5cm]

{\large Filip Hácha -- A16B0036P} \\[0.5cm]
{\large hachaf@students.zcu.cz}

% Bottom of the page
\vfill
{\large \today}
%\date{\today}

\end{center}
\end{titlepage}


\tableofcontents % obsah
\newpage

\section{Zadání}
\par Úkolem této práce bylo vytvořit webové stránky konference s libovolným tématem. Uživateli systému jsou autoři
příspěvků (vkládají abstrakty a PDF dokumenty), recenzenti příspěvků (hodnotí příspěvky) a administrátoři
(spravují uživatele, přiřazují příspěvky recenzentům a rozhodují o publikování příspěvků). Každý uživatel se bude do systému přihlašovat prostřednictvím uživatelského jména a hesla. Nepřihlášený uživatel vidí pouze publikované příspěvky.
Nový uživatel se bude může zaregistrovat, čímž získá status autora.
\par Přihlášený autor vidí svoje příspěvky a stav, ve kterém se nacházejí
(v recenzním řízení / přijat +hodnocení / odmítnut +hodnocení). Příspěvky může přidávat, editovat a volitelně i mazat.
Přihlášený recenzent vidí příspěvky, které mu byly přiděleny k recenzi, a může je hodnotit (alespoň 3 kritéria).
Pokud příspěvek nebyl dosud schválen, tak své hodnocení může změnit.
Administrátor spravuje uživatele (určuje jejich role a může uživatele zablokovat či smazat), přiřazuje neschválené příspěvky recenzentům k ohodnocení (každý příspěvek bude recenzován minimálně třemi recenzenty) a na základě recenzí rozhoduje o přijetí nebo odmítnutí příspěvku. Přijaté příspěvky jsou automaticky publikovány ve veřejné části webu.
\par Databáze musí obsahovat alespoň 3 tabulky dostatečně naplněné daty pro předvedení funkčnosti aplikace.

\section{Použité technologie}
\par Pro vytvoření databázové vrstvy programu byl použit systém MySQL Ver 15.1 Distribuce 10.1.26 - MariaDB.
\par Backend aplikační části je implementován pomocí jazyku PHP (podle zadání) a při vývoji aplikace byla použita verze 5.6.
\par Frontend aplikace je realizován za pomocí CSS frameworku Bootstrap v. 4, knihovny jQuery, kterou používá samotný
Bootstrap a zároveň byla použita pro některé dynamické části aplikace, a template engine knihovny Twig v. 2.4.5.

\section{Popis arhitektury}
\par Aplikace je strukturována ve smyslu návrhového vzoru MVC (Model View Controller) a je rozdělena do několika vrstev.

\section{Databáze}
\par Databáze aplikace je tvořena 4 tabulkami. Tabulka CONUSER obsahuje data o uživatelích konference, tabulka POST obsahuje příspěvky autorů,
tabulka REVIEW obsahuje hodnocení jednotlivých příspěvků a tabulka ASSIGNMENT přiřazuje jednotlivé příspěvky různým recenzentům.

\subsection{Model}
\par Datový model aplikace je reprezentován třídami PHP, které mají atributy odpovídající jednotlivým sloupcům
databázových tabulek a neposkytují žádnou další funkčnost, slouží pouze pro sdružování dat v aplikaci.

\subsection{Přístup k modelu}
\par Přístup k datům uložených v tabulkách mají na starosti třídy vrstvy, která zajišťujě připojení k databázi,
obsahuje funkce, které umožňůjí výběr, vkládání, mazání i úpravu dat v tabulkách a jako výsledky výběru poskytuje třídy modelu.

\subsection{Kontroléry}
\par Kontroléry mají na starosti předávání dat získaných z databáze do pohledů a zpracování požadavků a událostí,
které na jednotlivých pohledech uživatel provede, jako např. vytváření nových záznamů nebo úpravu již existujícíh
záznamů v databázi.

\subsection{Pohledy}
\par Pohledy jednotlivých usecasů aplikace jsou realizovány HTML soubory obsahující navíc příkazy template enginu Twig,
které umožňují v nich zobrazovat data poskytnutá kontroléry jako třídy modelu.

\subsection{Struktura adresáře}
\begin{itemize}
    \item{content} - Soubory CSS stylů, JavaScript soubory a obrázky, které aplikace využívá
    \item{controller} - PHP třídy kontrolerů
    \item{dbconnector} - Třídy zajíšťující propojení s databází
    \item{dbscripts} - SQL skripty pro založení databáze
    \item{documentation} - Adresář s dokumentací
    \item{inc} - PHP třídy s pomocnými funkcemi, konstantami a třídou, která zajišťující routování
    \item{model} - PHP třídy datového modelu
    \item{posts} - Nahrané PDF příspěvky
    \item{twig} - Šablonový systém Twig
    \item{vendor} - Soubory Bootstrapu, a další knihovny
    \item{view} - Šablony pohledů aplikace
\end{itemize}

\section{Verzování aplikace}
\par Pro verzování aplikace během vývoje byl použit repozitář systému Git na serveru GitHub. Repozitář je veřeně
přístupný na adrese \url{https://github.com/hachaf/KIV_WEB_CONFERENCE}.

\section{Závěr}
\par Vytvořená aplikace obsahuje funční usecasy popsané v zadání implementované v programovacím jazyce PHP a je
organizována podle návrhového vzoru MVC.


\appendix
\newpage

\end{document}